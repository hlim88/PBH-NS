\documentclass[prd]{revtex4}
\usepackage{amsmath, graphicx}
\usepackage{grffile}
\usepackage{dcolumn}
\usepackage{bm}
\usepackage{epsfig}
\usepackage{mathrsfs}  %  package for the "curly" fonts 
\usepackage{subfigure}
\usepackage{multirow}
\usepackage{epstopdf}
\usepackage{amsmath}
\usepackage{algorithmicx}
\usepackage{amssymb}
\usepackage{tensor}
\usepackage[math]{cellspace}
\usepackage{bookmark}


\newcommand*\apost{\textsc{\char13}}

\makeatletter
\renewcommand*\env@matrix[1][\arraystretch]{%
  \edef\arraystretch{#1}%
  \hskip -\arraycolsep
  \let\@ifnextchar\new@ifnextchar
  \array{*\c@MaxMatrixCols c}}
\makeatother


\topmargin 0.30in
\textheight 9.00in

 \addtolength{\voffset}{-2cm}
 
\begin{document}

\title{General Relativistic Hydrodynamical System in Spherically Symmetric Metric with Horizon Penetrating Coordinate}

\author{Hyun Lim}

%\pacs{}
\maketitle

We present a horizon penetrating coordinate for spherically symmetric metric. We adopt $G=c=1$ unit system. Consider usual Schwarzschild line element in BL coordinate for practice
\begin{align}
\label{eqn:schw_bl}
ds^2 = - \left(1- \frac{2M}{r} \right) dt^2 + \left(1- \frac{2M}{r} \right)^{-1} dr^2 + r^2 d\Omega^2 
\end{align}
where $M$ is a mass. Compare this with usual 3+1 line element form $ds^2 = - \alpha^2 dt^2 + \gamma_{ij} ( dx^2 + \beta^i dt)(dx^j + \beta^j dt)$ we can identify the lapse $\alpha^2 = \left(1- \frac{2M}{r} \right)$. As we know, the line element(Eqn.~\ref{eqn:schw_bl}) is singular at the horizon ($r=2M$) and lapse collapse to zero. This can be problematic because equations of motion for the metric can be become exponentially unstable in the presence of a coordinate singularity without some regularization technique.

One way to resolve this problem is to move to a horizon penetrating coordinate system where this singularity is not present. The Kerr-Schild coordinates are one such coordinate system. 

For example, Schwarzschild solution in spherical type Kerr-Schild coordinates
\begin{align}
\alpha &= \sqrt{\frac{r}{r+2M}} \\
\beta^r &= \frac{2M}{r+2M} \\
\beta_r &=\frac{2M}{r} \\
\beta^\theta &= \beta^\varphi = 0 \\
K_{ij} &= \textrm{diag} \left[ -\frac{2M(r+M)}{\sqrt{r^5 (r+2M)}} , 2M \sqrt{\frac{r}{r+2M}}, K_{\theta \theta} \sin^2 \theta \right]
\end{align}
Schwarzschild solution in Cartesian type Kerr-Schild coordinate
\begin{align}
\alpha &= \sqrt{\frac{r}{r+2M}} \\
\beta^i &= \frac{2M}{r} \frac{x^i}{r+2M} \\
\beta_i &=\frac{2M x_i}{r^2} \\
K_{ij} &= \-\frac{2M}{r^4} \sqrt{\frac{r}{r+2M}} \left[ \left( \frac{M}{r}+2 \right) x_i x_j - r^2 \delta_{ij} \right]
\end{align}
where $x^i = (x,y,z)$ which is usual spatial Cartesian coordinate. In both cases, we can see lapse is regular at the horizon.

General spherical symmetric line element in polar-areal form
\begin{align}
\label{eqn:ss-met-pa}
ds^2 = - \alpha(r)^2 dt^2 + a(r)^2 dr^2 + r^2 d \Omega^2
\end{align}
where $\alpha$ is referred as lapse function. Compare with above Schwarzschild solution, $\alpha = 1/a$. 

Now consider a transformation of the Schwarzschild time t coordinate to a new generic coordinate $\hat{t}$ according to
\begin{align}
d\hat{t} = dt + a^2 \sqrt{1-\frac{g}{a^2}} dr
\end{align}
where $g(r)$ is arbitrary function. Substitute this into $ds^2 = -\alpha^2 dt^2 + a^2 dr^2 + r^2 d\Omega^2$ gives
\begin{align}
ds^2 &= - \alpha^2 \left (d\hat{t} - a^2 \sqrt{1-\frac{g}{a^2}} dr\right)^2 + a^2 dr^2 + r^2 d\Omega^2 \nonumber \\
&= - \alpha^2 d \hat{t}^2 + 2 \sqrt{1-\frac{g}{a^2}} d\hat{t} dr + g dr^2 + r^2 d \Omega^2
\end{align}
Compare this with usual 3+1 framework
\begin{align}
ds^2 = - \alpha^2 d \hat{t}^2 + \gamma_{ij} (dx^i + \beta^i d\hat{t}\,)(dx^j + \beta^j d\hat{t}\,)
\end{align}
and so into the lapse $\alpha = 1/\sqrt{g}$, the shift $\beta_i = (\sqrt{1-g/a^2},0,0)$ or $\beta^i = \gamma^{ij} \beta_j$ and the spatial metric of the constant $\hat{t}$ hypersurface $\gamma_{ij} = diag(g,r^2,r^2\sin^2 \theta)$. 

If we choose $\alpha = \sqrt{1-2M/r} = 1/a$ and $g = 1+2M/r$ like in previous (which we will use this), we get
\begin{align}
ds^2 = - \left(1 - \frac{2M}{r} \right) d \hat{t}^2 + \frac{4M}{r} d \hat{t} dr + \left(1+\frac{2M}{r} \right) dr^2 + r^2 d\Omega^2
\end{align}
which is Schwarzschild in Kerr-Schild coordinate (or Eddington-Finkelstein coordinate). And correspondingly, $\alpha = \sqrt{r/(r+2M)}$, $\beta_i = (2M/r,0,0)$, and $\gamma_{ij} = diag(1+2M/r,r^2,r^2\sin^2 \theta)$ which are same as above.

As you can see here, the KS (or EF) form of the metric represents an analytic expansion of the Schwarzschild solution from the region $2M < r< \infty$ to  $0<r<\infty$. Thus, we apply this coordinate transformation for our equations.

It is good to rewrite the metric into usual $3+1$ variable form i.e. keep it geometric variables (should be careful of confusion) with considering time dependent case (This is almost same as Marsa and Choptuik\apost s paper). Here, we use $t$ for time coordinate that we used above.
\begin{align}
\label{eqn:gen-sph-met}
ds^2 = (-\alpha^2 + a^2 \beta^2) dt^2 + 2a^2 \beta dt dr + a^2 dr^2 + r^2 b^2 d \Omega^2
\end{align}
where $\alpha$, $a$, $b$, and $\beta$ are functions of $r$ and $t$, and $d\Omega^2$ is the metric of unit sphere. From this, we can calculate non-vanishing components of connection coefficients and Ricci tensors for $i,j$, and $k$ (spatial indices)
\begin{align}
&\Gamma\indices{^r_{rr}} = \frac{\partial_r a}{a}, \,\,\,\,\,\,\,\, \Gamma\indices{^r_{\theta \theta}} = - \frac{rb \partial_r (rb)}{a^2}, \,\,\,\,\,\,\,\, \Gamma\indices{^\theta_{r \theta}} = \frac{\partial_r (rb)}{rb} \nonumber  \\
&\Gamma\indices{^r_{\varphi \varphi}} = -\sin^2 \theta \frac{rb \partial_r (rb)}{a^2}, \,\,\,\,\,\,\,\, \Gamma\indices{^\varphi_{r \varphi}} = \Gamma\indices{^\theta_{r \theta}} \nonumber \\
&\Gamma\indices{^\theta_{\varphi \varphi}} = -\sin \theta \cos \theta, \,\,\,\,\,\,\,\, \Gamma\indices{^\varphi_{\varphi \theta}} = -\cot \theta \nonumber 
\end{align}
\begin{align}
R\indices{^r_r} &= -\frac{2}{arb} \partial_r \left(\frac{\partial_r (rb)}{a} \right) \\
R\indices{^\theta_\theta} &= \frac{1}{ar^2b^2} \left[a-\partial_r \left(\frac{rb \partial_r (rb)}{a} \right) \right] \\
\end{align}

%In terms of horizon penetrating coordinate, we can generalize it in 3+1 form
%\begin{align}
%\label{eqn:gen-sph-met}
%ds^2 = (-\alpha^2 + a^2 \beta^2) dt^2 + 2a^2 \beta dt dr + a^2 dr^2 + r^2 b^2 d \Omega^2
%\end{align}
%where $\alpha$, $a$, $b$, and $\beta$ are functions of $r$ and $t$, and $d\Omega^2$ is the metric of unit sphere. This is nothing but ingoing Eddington-Finkelstein coordinate system (IEF). Consider Schwarzschild again in IEF
%\begin{align}
%ds^2 = - \left( 1- \frac{2M}{r} \right) dV^2 + 2 dV dr + r^2 d \Omega^2
%\end{align}
%Define a timelike coordinate $t = V-r$ then metric becomes
%\begin{align}
%ds^2 = - \left( 1- \frac{2M}{r} \right) dt^2 + \frac{4M}{r} dt dr +  \left( 1+ \frac{2M}{r} \right) dr^2 + r^2 d \Omega^2
%\end{align}
%Compare this with Eqn.~\ref{eqn:gen-sph-met}, various following metric components can be found
%\begin{align}
%\alpha &= \sqrt{\frac{r}{r+2M}} \\
%\beta &=\frac{2M}{r+2M} \\
%a &=\sqrt{\frac{r+2M}{r}}
%\end{align}
%and so on. Note that we can also fix the spatial degree of coordinate freedom by introducing a shifting areal coordinate $R \equiv  r + f(t)$ where $f(t)$ is some undetermined function.  

\section{Theoretical Model}

Here, as a beginning set up, we first use the perfect fluid approximation for the matted model. So the stress-energy tensor takes form
\begin{align}
\label{eqn:EMtPF}
T_{ab} = (\rho + P) u_a u_b +P g_{ab} 
\end{align}
where $u^a(r,t)$ is the 4-velocity of a given perfect element, $P(r,t)$ is the isotropic pressure, $\rho(r,t) = \rho_0(r,t) (1+\epsilon(r,t))$ is the energy density, $\rho_0 (r,t)$ is the rest-mass energy density, and $\epsilon(r,t)$ is the specific internal energy.

The equations of motion for this case are derive from the local conservative equations for energy and baryon number such that
\begin{align}
\nabla_a T\indices{^a_b} &= 0 \\
\nabla_a (\rho_0 u^a) &= 0 
\end{align}

We are using HRSC so we would like to write this in terms of flux-conservative form such that
\begin{align}
\partial_t {\bf U} + \partial_i {\bf F^i} = {\bf \Psi}
\end{align}
where ${\bf U}$, 
\begin{align}
{\bf U} = 
\begin{pmatrix}
\sqrt{\gamma} W \rho_0 \\
\sqrt{\gamma} \alpha T\indices{^t_j} \\
\alpha^2 \sqrt{\gamma} T^{tt} - \sqrt{\gamma} W \rho_0
\end{pmatrix}
\end{align}

and ${\bf F^i}$
\begin{align}
{\bf F^i} = 
\begin{pmatrix}
\sqrt{\gamma} W \rho_0 v^i \\
\sqrt{\gamma} \alpha T\indices{^i_j} \\
\alpha^2 \sqrt{\gamma} T^{ti} - \sqrt{\gamma} W \rho_0 v^i
\end{pmatrix}
\end{align}

and ${\bf \psi}$
\begin{align}
{\bf \Psi} = 
\begin{pmatrix}
0\\
\frac{1}{2}\sqrt{\gamma} \alpha T\indices{^{ab}}g_{ab,j} \\
\alpha^2 \sqrt{\gamma} (T^{at} \partial_a \alpha - \Gamma\indices{^0_{ab}} T^{ab} \alpha )
\end{pmatrix}
\end{align}
where $W$ is Lorentz factor such that $W = \alpha u^t$ and $v^i = u^i/u^t$. Under our choice of system, $u^a = (u^t, u^r, 0, 0)$ so we can reduce
\begin{align}
\partial_t ( r^2 a b W \rho_0) + \partial_r (r^2 a b W \rho_0 v^r) &= 0 \\
\partial_t (r^2 a b \alpha T\indices{^t_r}) + \partial_r ( r^2 a b \alpha T\indices{^r_r}) &= \frac{1}{2} r^2 a b T^{ab} g_{ab,r} \\
\partial_t ( \alpha^2 r^2 a b T^{tt} - r^2 a b W \rho_0 ) + \partial_r(\alpha^2 r^2 a b T^{tr} - r^2 a b W \rho_0 v^r) &= \alpha r^2 ab (T^{at} \partial_a \alpha - \Gamma\indices{^0_{ab}} T^{ab} \alpha )
\end{align}


Here I omit $\sin$ term in the metric determinant because it will be cancelled out anyway

It is useful to define variables like below
\begin{align}
D &= \rho_0 ab W  \\
E &= \rho_0 h  W^2 -P \\
S &= \rho_0 h W^2 v\\
\tau &= E-D
\end{align}
And nonzero components of $T^{ab}$ which we are using
\begin{align}
T\indices{^t_t} &= -E \\
T\indices{^t_r} &= \frac{ab}{\alpha} S \\
T\indices{^r_r} &=Sv+P\\
T\indices{^\theta_\theta}&=T\indices{^\varphi_\varphi}=P
\end{align}
where we define fluid velocity in Eulerian observer
\begin{align}
v = \frac{ab}{\alpha} v^r = \frac{ab u^r}{\alpha u^t}
\end{align}
then also $W=1 / \sqrt{1-v^2}$ and $h=1+\epsilon+P/\rho_0$ which is specific enthalpy.  Then we can reduce
\begin{align}
\partial_t ( r^2 D) + \partial_r \left(\frac{r^2\alpha}{ab}D v\right) &= 0 \\
\partial_t (r^2 S) + \partial_r \left( \frac{r^2 \alpha}{ab}( E v + P)\right) &= \frac{1}{2} r^2 a b T^{ab} g_{ab,r} \\
\partial_t ( r^2\tau) + \partial_r\left(\frac{r^2\alpha}{ab}(S - D v)\right) &= \alpha r^2 ab (T^{at} \partial_a \alpha - \Gamma\indices{^0_{ab}} T^{ab} \alpha )
\end{align}
or in the form
\begin{align}
\partial_t {\bf u} + \frac{1}{r^2} \partial_r (X r^2 {\bf f}) = \psi
\end{align}
\begin{align}
{\bf u} = 
\begin{pmatrix}
D \\
S \\
\tau
\end{pmatrix}, \,\,\,\,
{\bf f} = 
\begin{pmatrix}
Dv \\
Ev + P \\
S - Dv
\end{pmatrix}, \,\,\,\,
{\bf \psi} = 
\begin{pmatrix}
0 \\
\frac{ab}{2} a b T^{ab} g_{ab,r} \\
\alpha ab (T^{at} \partial_a \alpha - \Gamma\indices{^0_{ab}} T^{ab} \alpha )
\end{pmatrix}
\end{align}
where $X= \alpha / (ab)$ which is purely geometric factor. Some detail evaluation of RHS source terms are in \href{https://github.com/hlim88/PBH-NS/tree/master/tools}{here}






\iffalse
First, we define variables. 
%\begin{align}
%{\bf q} = \begin{bmatrix}
%D\\
%\Pi\\
%\Phi
%\end{bmatrix} , \,\,\,\,\,\,\,
%{\bf f} = \begin{bmatrix}
%Dv\\
%v(\Pi+P) + P\\
%v(\Phi + P) + P
%\end{bmatrix} , \,\,\,\,\,\,\,
%{\bf \psi} = \begin{bmatrix}
%0\\
%\Sigma\\
%-\Sigma
%\end{bmatrix} 
%\end{align}
%where $v$ is Eulerian velocity of fluid such that $ v = a u^r / (\alpha u^t)$, $X=\alpha/a$ is a purely geometric quantity and
%\begin{align}
%D &= a \rho_0 W\\
%\Pi &= E-D+S\\
%\Phi &=E-D-S\\
%S &= \rho_0 h W^2 v \\
%E &=\rho_0 h W^2 - P 
%\end{align}
where $W$ is Lorentz factor such that $W = \alpha u^t=1/\sqrt{1-v^2}$ with fluid velocity $v =(a u^r) / (\alpha u^t)$ and $h=1+\epsilon+P/\rho_0$ which is specific enthalpy. In our case, $u^a = (u^t, u^r, 0, 0)$

And nonzero components of $T^{ab}$ which we are using
\begin{align}
T\indices{^t_t} &= -E \\
T\indices{^t_r} &= \frac{a}{\alpha} S \\
T\indices{^r_r} &=Sv+P\\
T\indices{^\theta_\theta}&=T\indices{^\varphi_\varphi}=P
\end{align}

Using these variables under the metric which we consider, $\nabla_a (\rho_0 u^a)=0$ (continuity equation) gives
\begin{align}
\partial_t (\rho_0 u^t) + \partial_r (\rho_0 u^r) + \Gamma\indices{^t_{tt}}(\rho_0 u^t) + \Gamma\indices{^t_{tr}}(\rho_0 u^r) = 0
\end{align} 
In terms of our variables, $\rho_0 u^t = D/\alpha$, $\rho_0 u^r = D v /a$ so
\begin{align}
\partial_t (D/\alpha) + \partial_r (D/\alpha) + \Gamma\indices{^t_{tt}}(D/\alpha) + \Gamma\indices{^t_{tr}}(D v / a) = 0
\end{align} 

$\nabla_a T\indices{^a_b}=0$ gives
\begin{align}
\partial_a T\indices{^a_b} + \Gamma\indices{^a_{ac}}T\indices{^c_b} - \Gamma\indices{^c_{ab}}T\indices{^a_c} = 0
\end{align}
The covariant $t$-component of above equation gives energy equation $T\indices{^t_t} = - E$
\begin{align}
&\partial_t T\indices{^t_t} + \Gamma\indices{^t_{tc}}T\indices{^c_t} - \Gamma\indices{^c_{tt}}T\indices{^t_c} = 0 \nonumber \\
&\rightarrow \partial_t T\indices{^t_t} + \Gamma\indices{^t_{tt}}T\indices{^t_t} +\Gamma\indices{^t_{tr}}T\indices{^r_t}- \Gamma\indices{^t_{tt}}T\indices{^t_t} - \Gamma\indices{^r_{tt}}T\indices{^t_r} = 0 \nonumber \\
&\rightarrow \partial_t E + \Gamma\indices{^t_{tr}} \frac{a^3}{\alpha^3} S + \Gamma\indices{^r_{tt}} \frac{a}{\alpha} S = 0
\end{align}
Next, consider the covariant $r$-component 
\begin{align}
&\partial_t T\indices{^t_r} + \Gamma\indices{^t_{tc}}T\indices{^c_r} - \Gamma\indices{^c_{tr}}T\indices{^t_c} = 0 \nonumber \\
&\rightarrow \partial_t T\indices{^t_r} + \Gamma\indices{^t_{tt}}T\indices{^t_r} +\Gamma\indices{^t_{tr}}T\indices{^r_r}- \Gamma\indices{^t_{tr}}T\indices{^t_t} - \Gamma\indices{^r_{tr}}T\indices{^t_r} = 0 \nonumber \\
&\rightarrow \partial_t \left(\frac{a}{\alpha} S\right) + \Gamma\indices{^t_{tt}}(\frac{a}{\alpha} S)  +\Gamma\indices{^t_{tr}}(Sv+P) + \Gamma\indices{^t_{tr}}E - \Gamma\indices{^r_{tr}}\frac{a}{\alpha} S= 0 \nonumber 
\end{align}
Non-vanishing connection coefficients are evaluated via Mathematica. You can find it \href{https://github.com/hlim88/PBH-NS/tree/master/tools}{here}. After all simplifications, we have
\fi

%Here, we use the fact that $\sqrt{-g} = \alpha \sqrt{\gamma}$ where $\gamma = det(\gamma_{ij})$ which came from above metric.

Now consider the Einstein\apost s equations. Define below quantities that are appearing in the 3+1 equations
\begin{align}
\rho_{hydro} &= n_a n_b T^{ab} = \rho_0 h W^2 - P \\
S_i^{hydro} &= - \gamma_{ia} n_b T^{ab}  = \rho_0 h W u_i \\
S_{ij}^{hydro} &= \gamma_{ia} \gamma_{ib} T^{ab} = P \gamma_{ij} + \rho_0 h u_i u_j \\
S_{hydro} &= \gamma^{ij} S_{ij} = 3 P + \rho_0 h (W^2 -1)
\end{align}

The Einstein\apost s equations in the ADM form are
\begin{align}
\label{eqn:adm:gam}
\partial_t \gamma_{ij} &= - 2 \alpha K_{ij} + D_i \beta_j + D_j \beta_i \\ 
\label{eqn:adm:K}
\partial_t K\indices{^i_j} &= \alpha (R\indices{^i_j} + K K\indices{^i_j}) - D^i D_j \alpha - 8 \pi \alpha \left(S\indices{^i_j} - \frac{1}{2} \delta\indices{^i_j} (S-\rho)\right) \nonumber \\
& + \beta^k \partial_k K\indices{^i_j} + K\indices{^i_k} \partial_j \beta^k - K\indices{^k_j} \partial_k \beta^i
\end{align}
where $D_i$ is covariant derivative on spatial hypersurface. Momentum and Hamiltonian constraints are
\begin{align}
R+K^2 - K_{ij} K^{ij} = 16 \pi \rho \\
D_i K\indices{^i_j} - D_j K = 8 \pi S_j
\end{align}

Substitute hydro source terms ($\rho$, $S$ etc) from above then we have
\begin{align}
\partial_t K\indices{^i_j} &= \alpha (R\indices{^i_j} + K K\indices{^i_j}) - \gamma^{ik} (\partial_i \partial_k \alpha - \Gamma\indices{^l_{ik}} \partial_l \alpha) - 8 \pi \alpha \left(\frac{1}{2} \delta\indices{^i_j} (\rho_0 h - 2 P) + \rho_0 h u^i u_j \right) \nonumber \\
& + \beta^k \partial_k K\indices{^i_j} + K\indices{^i_k} \partial_j \beta^k - K\indices{^k_j} \partial_k \beta^i \\
\partial_t \gamma_{ij} &= - 2 \alpha K_{ij} + D_i \beta_j + D_j \beta_i \\ 
&R+K^2 - K_{ij} K^{ij} = 16 \pi (\rho_0 h W^2 -P)\\
&D_i K\indices{^i_j} - D_j K = 8 \pi \rho_0 h W u_j
\end{align}

From our choice of metric/coordinate system, we calculated non-trivial connection coefficients and Ricci tensors. Also, metric form suggests that $\beta^i = (\beta^r,0,0)$, $K\indices{^i_j} = diag(K\indices{^r_r}, K\indices{^\theta_\theta},K\indices{^\theta_\theta})$. Using these facts, the evolution equations for geometric quantities are
\begin{align}
\partial_t a &= - \alpha a K\indices{^r_r} + \partial_r (a \beta^r) \\
\partial_t b &= - \alpha b K\indices{^\theta_\theta} + \frac{\beta^r}{r} \partial_r (r \beta^r) \\
\partial_t K\indices{^r_r} &= \beta^r \partial_r K\indices{^r_r} + \alpha K\indices{^r_r} K - \frac{1}{a} \partial_r \left( \frac{\partial_r \alpha}{a} \right) - \frac{2 \alpha}{arb} \partial_r \left( \frac{\partial_r (rb)}{a} \right) \nonumber \\
& - 4 \pi \alpha \left[ (1+2 u^r u_r) \rho_0 h - 2P \right] \\
\partial_t K\indices{^\theta_\theta} &= \beta^r \partial_r K\indices{^\theta_\theta} + \alpha K\indices{^\theta_\theta} K - \frac{\alpha}{r^2 b^2} - \frac{1}{ar^2b^2} \partial_r \left( \frac{\alpha r b \partial_r (rb)}{a} \right) \nonumber \\
& - 4 \pi \alpha (\rho_0 h -2 P)
\end{align}
From constraints
\begin{align}
 \frac{1}{ar^2b^2} \left[a-\partial_r \left(\frac{rb \partial_r (rb)}{a} \right) \right] -\frac{2}{arb} \partial_r \left(\frac{\partial_r (rb)}{a} \right) + 2K\indices{^\theta_\theta}(K\indices{^\theta_\theta}+2 K\indices{^r_r}) &= 16 \pi (\rho_0 h W^2 -P) \nonumber \\
 \frac{\partial_t (r b)}{rb} (K\indices{^\theta_\theta} - K\indices{^r_r}) - \partial_r K\indices{^\theta_\theta} &= 4 \pi \rho_0 h W u_r
\end{align}


We can apply different choice of slicing (i.e. gauge choice) to reduce/determine above system. Possible (or simple) choices would be maximal or polar slicing. 

\subsection{Choice of Gauge}

\subsubsection{Maximal Slicing}
First, we consider maximal slicing i.e. $K=\partial_t K = 0$ then 

\begin{align}
\partial_t a &= - \alpha a K\indices{^r_r} + \partial_r (a \beta^r) \\
\partial_t b &=  \frac{\alpha b}{2} K\indices{^r_r} + \frac{\beta^r}{r} \partial_r (r \beta^r) \\
\partial_t K\indices{^r_r} &= \beta^r \partial_r K\indices{^r_r}  - \frac{1}{a} \partial_r \left( \frac{\partial_r \alpha}{a} \right) - \frac{2 \alpha}{arb} \partial_r \left( \frac{\partial_r (rb)}{a} \right) 
-  \frac{2 \alpha}{r^2 b^2} - \frac{2}{ar^2b^2} \partial_r \left( \frac{\alpha r b \partial_r (rb)}{a} \right)  \nonumber \\
& - 8 \pi \alpha \left[ (2+2 u^r u_r) \rho_0 h - 4P \right] 
\end{align}
From constraints
\begin{align}
 \frac{1}{ar^2b^2} \left[a-\partial_r \left(\frac{rb \partial_r (rb)}{a} \right) \right] -\frac{2}{arb} \partial_r \left(\frac{\partial_r (rb)}{a} \right) - \frac{3}{2}(K\indices{^r_r})^2 &= 16 \pi (\rho_0 h W^2 -P) \nonumber \\
  \partial_r K\indices{^r_r} -\frac{3 \partial_t (r b)}{rb}  K\indices{^r_r} &= 8 \pi \rho_0 h W u_r
\end{align}
Fluid EOM parts are same as previous

For lapse, we use 
\begin{align}
\partial_t K = - D^2 \alpha + \alpha (K^{ij}K_{ij} + 4 \pi (\rho + S))+ \beta^i D_i K
\end{align}
In our choice of gauge, this can be reduced
\begin{align}
D^2 \alpha &= \alpha ( K^{ij} K_{ij} + 4 \pi (\rho + S)) = \alpha \left( K^{ij} K_{ij} + 8 \pi \left[P + \rho_0 h \left(W^2 - \frac{1}{2}  \right)\right]\right) \nonumber \\
&= \alpha \left( 2(K\indices{^r_r})^2 + 8 \pi \left[P + \rho_0 h \left(W^2 - \frac{1}{2}  \right)\right]\right)
\end{align}
Also, for shift, we use
\begin{align}
\partial_t \ln \sqrt{\gamma} = - \alpha K + D_i \beta^i 
\end{align}
In maximal slicing, this reduces
\begin{align}
D_i \beta^i = - \partial_t \ln \sqrt{\gamma}  
\end{align}
or we can write 
\begin{align}
\partial_i \beta^i = - \partial_t \sqrt{\gamma}  
\end{align}
This shows that the proper volume element $\sqrt{\gamma}$ satisfies a continuity equation in maximal slicing.

In terms of our metric choice and variable
\begin{align}
\partial_r \beta^r = \frac{b}{2b+\beta^r} \left[\frac{\alpha K\indices{^r_r}}{2} - \frac{(\beta^r)^2 b}{r}- \frac{\partial_r a}{a} \beta^r \right]
\end{align}

\subsubsection{Spherical-BSSN with 1+log and $\Gamma$ driver}
Recall the eqns
\begin{align}
\partial_t a &= - \alpha a K\indices{^r_r} + \partial_r (a \beta^r) \\
\partial_t b &=  \frac{\alpha b}{2} K\indices{^r_r} + \frac{\beta^r}{r} \partial_r (r \beta^r) \\
\partial_t K\indices{^r_r} &= \beta^r \partial_r K\indices{^r_r}  - \frac{1}{a} \partial_r \left( \frac{\partial_r \alpha}{a} \right) - \frac{2 \alpha}{arb} \partial_r \left( \frac{\partial_r (rb)}{a} \right) 
-  \frac{2 \alpha}{r^2 b^2} - \frac{2}{ar^2b^2} \partial_r \left( \frac{\alpha r b \partial_r (rb)}{a} \right)  \nonumber \\
& - 8 \pi \alpha \left[ (2+2 u^r u_r) \rho_0 h - 4P \right] 
\end{align}
From constraints
\begin{align}
 \frac{1}{ar^2b^2} \left[a-\partial_r \left(\frac{rb \partial_r (rb)}{a} \right) \right] -\frac{2}{arb} \partial_r \left(\frac{\partial_r (rb)}{a} \right) - \frac{3}{2}(K\indices{^r_r})^2 &= 16 \pi (\rho_0 h W^2 -P) \nonumber \\
  \partial_r K\indices{^r_r} -\frac{3 \partial_t (r b)}{rb}  K\indices{^r_r} &= 8 \pi \rho_0 h W u_r
\end{align}
Let \apost s consider usual BSSN form (make conformal transformation $\hat{\gamma} \rightarrow e^{4\chi} \gamma$

%Another possible choice to set the lapse is demanding that the ingoing combination of tangent vectors $\vec{\partial}_t - \vec{\partial}_r$ be null. This gives a condition on the metric : $g_{tt} - 2 g_{tr} + g_{rr}=0$. This gives $\alpha = a(1-\beta)$


%Further, a sufficient set of Einstein\apost s equations for geometric variable $\alpha$ and $a$ are given by the nontrivial component of momentum constraint
%\begin{align}
%\partial_t a = - 4 \pi r \alpha a^2 S
%\end{align}
%and by the polar slicing condition which follows from the demand that metric have the spherically symmetric form for all time
%\begin{align}
%\partial_r (\ln \alpha) = a^2 \left[4 \pi r (Sv+P) + \frac{m}{r^2} \right]
%\end{align}
%and from Hamiltonian constraint
%\begin{align}
%\partial_r a = a^3 \left(4 \pi r E - \frac{m}{r^2} \right)
%\end{align}

%To solve these sets of equations on Schwarzschild background in KS, we consider following coordinate transformation in time 
%\begin{align}
%\hat{t} = t + 2 M \ln | \frac{r}{2M} - 1 | + k
%\end{align}
%where $k$ is arbitrary constant. So for arbitrary function $F$
%\begin{align}
%\frac{\partial F}{\partial t} = \frac{\partial \hat{t}}{\partial t} \frac{\partial F}{\partial \hat{t}} = \left(1+\frac{2MXv}{r-2M} \right) \frac{\partial F}{\partial \hat{t}} = \left(1+\frac{2Mv}{r} \right) \frac{\partial F}{\partial \hat{t}}
%\end{align}
%We apply this rule to above equations for HPC


%From the metric,
%\begin{align}
%ds^2 = - \alpha^2 dt^2 + a^2 dr^2 + r^2 d \Omega^2
%\end{align}
%Identify $a = \sqrt{r/(r-2M)}$ which is usual Schwarzschild in Schwarzschild coordinate then Einstein\apost s equations and energy conservation imply the TOV system of ODEs
%\begin{align}
%\frac{d m}{dr} &= 4 \pi r^2 \rho \\
%\frac{d P}{dr} &= - (\rho + P) \frac{M + 4 \pi r^3 P}{r(r-2M)} \\
%\frac{d (\ln \alpha)}{d r} &= \frac{M+4\pi r^3 P}{r(r-2M)}
%\end{align}
%Again $\rho = \rho_0 ( 1+ \epsilon)$. Now consider this system in Kerr-Schild coordinate which provides horizon penetrating. Metric becomes from previous section
%\begin{align}
%ds^2 = (- \alpha^2 + a^2 \beta^2) dt^2 + 2 a^2 \beta dt dr + a^2 dr^2 + r^2 d \Omega^2
%\end{align}
%In this case, $\alpha$, $a$, and $\beta$ are function of $r$. Also we define 



%obtaining TOV equation. Here, we still have spherical symmetry and staticity so
%\begin{align}
%\frac{d}{dr} ( \sqrt{-g} \rho_0 u^1) &= 0 \nonumber \\
%\frac{d}{dr} ( \sqrt{-g} T^1_0 ) &= 0 \nonumber 
%\end{align}
%where $T_{ab}=(\rho + P) u_a u_b + Pg_{ab}$ or $T_{ab}=\rho_0 h u_a u_b + Pg_{ab}$ where $h = 1+\epsilon+p/\rho_0$ which is specific entalpy

\subsection{Initial Data}
Our initial NS model is approximated by solution of TOV. After the initial data calculation, an in-going velocity profile is added to drive the star to collapse. We follow the way is described in (https://arxiv.org/pdf/gr-qc/0107045.pdf). First, specifying the coordinate velocity
\begin{align}
U \equiv \frac{dr}{dt} = \frac{u^r}{u^r}
\end{align}
of the star. In general, the profile take the algebraic form $U_g(x) = A_0 (x^3 - B_0 x)$ where $x \equiv  r/R_{star}$ and $R_{star}$ is the radius of the TOV solution.

In this work, we set two profiles
\begin{align}
U(x) =
\begin{cases}
U_1(x) = U_{crit} ( x^3 - 3x) & x < x_{tlv} \\
\\
U_2(x) = 0 & \textrm{otherwise} \\
\end{cases}
\end{align}
$U_{crit}$ is the amplitude that occurs critical collapse, and $x_{tlv}$ the region that forms black hole. 

Our interest is interaction between BH inside of NS we set $x_{tlv}$ is small value such as 1\% of size of star i.e. $x_{tlv} = 0.01$ since $x$ is normalized radius by star radius ($x_{tlv}$ must be smaller than $1$).

\subsection{Analytic Case}

For code test and validation, we use well-known Michel problem

\end{document}
