\documentclass[prd]{revtex4}
\usepackage{amsmath, graphicx}
\usepackage{grffile}
\usepackage{dcolumn}
\usepackage{bm}
\usepackage{epsfig}
\usepackage{mathrsfs}  %  package for the "curly" fonts 
\usepackage{subfigure}
\usepackage{multirow}
\usepackage{epstopdf}
\usepackage{amsmath}
\usepackage{algorithmicx}
\usepackage{amssymb}
\usepackage{tensor}
\usepackage[math]{cellspace}
\usepackage{bookmark}


\newcommand*\apost{\textsc{\char13}}

\makeatletter
\renewcommand*\env@matrix[1][\arraystretch]{%
  \edef\arraystretch{#1}%
  \hskip -\arraycolsep
  \let\@ifnextchar\new@ifnextchar
  \array{*\c@MaxMatrixCols c}}
\makeatother


\topmargin 0.30in
\textheight 9.00in

 \addtolength{\voffset}{-2cm}
 
\begin{document}

\title{Horizon Penetrating Coordinate for Spherically Symmetric Metric}

\author{Hyun Lim}

%\pacs{}
\maketitle

We present a horizon penetrating coordinate for spherically symmetric metric. We adopt $G=c=1$ unit system. Consider usual Schwarzschild line element in BL coordinate for practice
\begin{align}
\label{eqn:schw_bl}
ds^2 = - \left(1- \frac{2M}{r} \right) dt^2 + \left(1- \frac{2M}{r} \right)^{-1} dr^2 + r^2 d\Omega^2 
\end{align}
where $M$ is a mass. Compare this with usual 3+1 line element form $ds^2 = - \alpha^2 dt^2 + \gamma_{ij} ( dx^2 + \beta^i dt)(dx^j \beta^j dt)$ we can identify the lapse $\alpha^2 = \left(1- \frac{2M}{r} \right)$. As we know, the line element(Eqn.~\ref{eqn:schw_bl}) is singular at the horizon ($r=2M$) and lapse collapse to zero. This can be problematic because equations of motion for the metric can be become exponentially unstable in the presence of a coordinate singularity without some regularization technique.

One way to resolve this problem is to move to a horizon penetrating coordinate system where this singularity is not present. The Kerr-Schild coordinates are one such coordinate system. General spherical symmetric line element in polar-areal form
\begin{align}
\label{eqn:ss-met-pa}
ds^2 = - \alpha(r,t)^2 dt^2 + a(r,t)^2 dr^2 + r^2 d \Omega^2
\end{align}
where $\alpha$ is referred as lapse function. Compare with above Schwarzschild solution, $\alpha = 1/a$.

\section{Model}

Here, we first use the perfect fluid approximation for the matted model of our neutron stars. So the stress-energy tensor takes form
\begin{align}
\label{eqn:EMtPF}
T_{ab} = (\rho + P) u_a u_b +P g_{ab} 
\end{align}




\end{document}
