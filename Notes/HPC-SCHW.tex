\documentclass[prd]{revtex4}
\usepackage{amsmath, graphicx}
\usepackage{grffile}
\usepackage{dcolumn}
\usepackage{bm}
\usepackage{epsfig}
\usepackage{mathrsfs}  %  package for the "curly" fonts 
\usepackage{subfigure}
\usepackage{multirow}
\usepackage{epstopdf}
\usepackage{amsmath}
\usepackage{algorithmicx}
\usepackage{amssymb}
\usepackage{tensor}
\usepackage[math]{cellspace}
\usepackage{bookmark}


\newcommand*\apost{\textsc{\char13}}

\makeatletter
\renewcommand*\env@matrix[1][\arraystretch]{%
  \edef\arraystretch{#1}%
  \hskip -\arraycolsep
  \let\@ifnextchar\new@ifnextchar
  \array{*\c@MaxMatrixCols c}}
\makeatother


\topmargin 0.30in
\textheight 9.00in

 \addtolength{\voffset}{-2cm}
 
\begin{document}

\title{Horizon Penetrating Coordinate for Spherically Symmetric Metric}

\author{Hyun Lim}

%\pacs{}
\maketitle

We present a horizon penetrating coordinate for spherically symmetric metric. We adopt $G=c=1$ unit system. Consider usual Schwarzschild line element in BL coordinate for practice
\begin{align}
\label{eqn:schw_bl}
ds^2 = - \left(1- \frac{2M}{r} \right) dt^2 + \left(1- \frac{2M}{r} \right)^{-1} dr^2 + r^2 d\Omega^2 
\end{align}
where $M$ is a mass. Compare this with usual 3+1 line element form $ds^2 = - \alpha^2 dt^2 + \gamma_{ij} ( dx^2 + \beta^i dt)(dx^j + \beta^j dt)$ we can identify the lapse $\alpha^2 = \left(1- \frac{2M}{r} \right)$. As we know, the line element(Eqn.~\ref{eqn:schw_bl}) is singular at the horizon ($r=2M$) and lapse collapse to zero. This can be problematic because equations of motion for the metric can be become exponentially unstable in the presence of a coordinate singularity without some regularization technique.

One way to resolve this problem is to move to a horizon penetrating coordinate system where this singularity is not present. The Kerr-Schild coordinates are one such coordinate system. 

For example, Schwarzschild solution in spherical type Kerr-Schild coordinates
\begin{align}
\alpha &= \sqrt{\frac{r}{r+2M}} \\
\beta^r &= \frac{2M}{r+2M} \\
\beta_r &=\frac{2M}{r} \\
\beta^\theta &= \beta^\varphi = 0 \\
K_{ij} &= \textrm{diag} \left[ -\frac{2M(r+M)}{\sqrt{r^5 (r+2M)}} , 2M \sqrt{\frac{r}{r+2M}}, K_{\theta \theta} \sin^2 \theta \right]
\end{align}
Schwarzschild solution in Cartesian type Kerr-Schild coordinate
\begin{align}
\alpha &= \sqrt{\frac{r}{r+2M}} \\
\beta^i &= \frac{2M}{r} \frac{x^i}{r+2M} \\
\beta_i &=\frac{2M x_i}{r^2} \\
K_{ij} &= \-\frac{2M}{r^4} \sqrt{\frac{r}{r+2M}} \left[ \left( \frac{M}{r}+2 \right) x_i x_j - r^2 \delta_{ij} \right]
\end{align}
where $x^i = (x,y,z)$ which is usual spatial Cartesian coordinate. In both cases, we can see lapse is regular at the horizon.

General spherical symmetric line element in polar-areal form
\begin{align}
\label{eqn:ss-met-pa}
ds^2 = - \alpha(r)^2 dt^2 + a(r)^2 dr^2 + r^2 d \Omega^2
\end{align}
where $\alpha$ is referred as lapse function. Compare with above Schwarzschild solution, $\alpha = 1/a$. 

Now consider a transformation of the Schwarzschild time t coordinate to a new generic coordinate $\hat{t}$ according to
\begin{align}
d\hat{t} = dt + a^2 \sqrt{1-\frac{g}{a^2}} dr
\end{align}
where $g(r)$ is arbitrary function. Substitute this into $ds^2 = -\alpha^2 dt^2 + a^2 dr^2 + r^2 d\Omega^2$ gives
\begin{align}
ds^2 &= - \alpha^2 \left (d\hat{t} - a^2 \sqrt{1-\frac{g}{a^2}} dr\right)^2 + a^2 dr^2 + r^2 d\Omega^2 \nonumber \\
&= - \alpha^2 d \hat{t}^2 + 2 \sqrt{1-\frac{g}{a^2}} d\hat{t} dr + g dr^2 + r^2 d \Omega^2
\end{align}
Compare this with usual 3+1 framework
\begin{align}
ds^2 = - \alpha^2 d \hat{t}^2 + \gamma_{ij} (dx^i + \beta^i d\hat{t}\,)(dx^j + \beta^j d\hat{t}\,)
\end{align}
and so into the lapse $\alpha = 1/\sqrt{g}$, the shift $\beta_i = (\sqrt{1-g/a^2},0,0)$ or $\beta^i = \gamma^{ij} \beta_j$ and the spatial metric of the constant $\hat{t}$ hypersurface $\gamma_{ij} = diag(g,r^2,r^2\sin^2 \theta)$. 

If we choose $\alpha = \sqrt{1-2M/r} = 1/a$ and $g = 1+2M/r$ like in previous (which we will use this), we get
\begin{align}
ds^2 = - \left(1 - \frac{2M}{r} \right) d \hat{t}^2 + \frac{4M}{r} d \hat{t} dr + \left(1+\frac{2M}{r} \right) dr^2 + r^2 d\Omega^2
\end{align}
which is Schwarzschild in Kerr-Schild coordinate (or Eddington-Finkelstein coordinate). And correspondingly, $\alpha = \sqrt{r/(r+2M)}$, $\beta_i = (2M/r,0,0)$, and $\gamma_{ij} = diag(1+2M/r,r^2,r^2\sin^2 \theta)$ which are same as above.

As you can see here, the KS (or EF) form of the metric represents an analytic expansion of the Schwarzschild solution from the region $2M < r< \infty$ to  $0<r<\infty$. Thus, we apply this coordinate transformation for our equations.

%In terms of horizon penetrating coordinate, we can generalize it in 3+1 form
%\begin{align}
%\label{eqn:gen-sph-met}
%ds^2 = (-\alpha^2 + a^2 \beta^2) dt^2 + 2a^2 \beta dt dr + a^2 dr^2 + r^2 b^2 d \Omega^2
%\end{align}
%where $\alpha$, $a$, $b$, and $\beta$ are functions of $r$ and $t$, and $d\Omega^2$ is the metric of unit sphere. This is nothing but ingoing Eddington-Finkelstein coordinate system (IEF). Consider Schwarzschild again in IEF
%\begin{align}
%ds^2 = - \left( 1- \frac{2M}{r} \right) dV^2 + 2 dV dr + r^2 d \Omega^2
%\end{align}
%Define a timelike coordinate $t = V-r$ then metric becomes
%\begin{align}
%ds^2 = - \left( 1- \frac{2M}{r} \right) dt^2 + \frac{4M}{r} dt dr +  \left( 1+ \frac{2M}{r} \right) dr^2 + r^2 d \Omega^2
%\end{align}
%Compare this with Eqn.~\ref{eqn:gen-sph-met}, various following metric components can be found
%\begin{align}
%\alpha &= \sqrt{\frac{r}{r+2M}} \\
%\beta &=\frac{2M}{r+2M} \\
%a &=\sqrt{\frac{r+2M}{r}}
%\end{align}
%and so on. Note that we can also fix the spatial degree of coordinate freedom by introducing a shifting areal coordinate $R \equiv  r + f(t)$ where $f(t)$ is some undetermined function.  

\section{Theoretical Model}

Here, as a beginning set up, we first use the perfect fluid approximation for the matted model. So the stress-energy tensor takes form
\begin{align}
\label{eqn:EMtPF}
T_{ab} = (\rho + P) u_a u_b +P g_{ab} 
\end{align}
where $u^a(r,t)$ is the 4-velocity of a given perfect element, $P(r,t)$ is the isotropic pressure, $\rho(r,t) = \rho_0(r,t) (1+\epsilon(r,t))$ is the energy density, $\rho_0 (r,t)$ is the rest-mass energy density, and $\epsilon(r,t)$ is the specific internal energy.

The equations of motion for this case are derive from the local conservative equations for energy and baryon number such that
\begin{align}
\nabla_a T\indices{^a_b} &= 0 \\
\nabla_a (\rho_0 u^a) &= 0 
\end{align}

We follow usual remaining set-up i.e. employ Euler velocity, using Primitive variables etc for our system

\subsection{GR-Hydro Equation in HPC}

Here we solve GR hydrodynamic equations on Schwarzschild background in KS coordinate system. We follow standard techniques that describe in many literatures. Fluid EOM conservative form 
\begin{align}
\partial_t {\bf q} +\frac{1}{r^2} \partial_r (r^2 X {\bf f}) = {\bf \psi}
\end{align}
where
\begin{align}
{\bf q} = \begin{bmatrix}
D\\
\Pi\\
\Phi
\end{bmatrix} , \,\,\,\,\,\,\,
{\bf f} = \begin{bmatrix}
Dv\\
v(\Pi+P) + P\\
v(\Phi + P) + P
\end{bmatrix} , \,\,\,\,\,\,\,
{\bf \psi} = \begin{bmatrix}
0\\
\Sigma\\
-\Sigma
\end{bmatrix} 
\end{align}
where $v$ is Eulerian velocity of fluid such that $ v = a u^t / (\alpha u^r)$ and
\begin{align}
D &= a \rho_0 W\\
\Pi &= E-D+S\\
\Phi &=E-D-S\\
S &= \rho_0 h W^2 v \\
E &=\rho_0 h W^2 - P 
\end{align}
where $W$ is Lorentz factor such that $W = \alpha u^t=1/\sqrt{1-v^2}$ and $h=1+\epsilon+P/\rho_0$ which is specific enthalpy. 

Further, a sufficient set of Einstein\apost s equations for geometric variable $\alpha$ and $a$ are given by the nontrivial component of momentum constraint
\begin{align}
\partial_t a = - 4 \pi r \alpha a^2 S
\end{align}
and by the polar slicing condition which follows from the demand that metric have the spherically symmetric form for all time
\begin{align}
\partial_r (\ln \alpha) = a^2 \left[4 \pi r (Sv+P) + \frac{m}{r^2} \right]
\end{align}
and from Hamiltonian constraint
\begin{align}
\partial_r a = a^3 \left(4 \pi r E - \frac{m}{r^2} \right)
\end{align}
To solve these sets of equations on Schwarzschild background in KS, we consider following coordinate transformation in time 
\begin{align}
\hat{t} = t + 2 M \ln | \frac{r}{2M} - 1 | + k
\end{align}
where $k$ is arbitrary constant. So for arbitrary function $F$
\begin{align}
\frac{\partial F}{\partial t} = \frac{\partial \hat{t}}{\partial t} \frac{\partial F}{\partial \hat{t}} = \left(1+\frac{2Mv}{r-2M} \right) \frac{\partial F}{\partial \hat{t}}
\end{align}
We apply this rule to above equations for HPC


%From the metric,
%\begin{align}
%ds^2 = - \alpha^2 dt^2 + a^2 dr^2 + r^2 d \Omega^2
%\end{align}
%Identify $a = \sqrt{r/(r-2M)}$ which is usual Schwarzschild in Schwarzschild coordinate then Einstein\apost s equations and energy conservation imply the TOV system of ODEs
%\begin{align}
%\frac{d m}{dr} &= 4 \pi r^2 \rho \\
%\frac{d P}{dr} &= - (\rho + P) \frac{M + 4 \pi r^3 P}{r(r-2M)} \\
%\frac{d (\ln \alpha)}{d r} &= \frac{M+4\pi r^3 P}{r(r-2M)}
%\end{align}
%Again $\rho = \rho_0 ( 1+ \epsilon)$. Now consider this system in Kerr-Schild coordinate which provides horizon penetrating. Metric becomes from previous section
%\begin{align}
%ds^2 = (- \alpha^2 + a^2 \beta^2) dt^2 + 2 a^2 \beta dt dr + a^2 dr^2 + r^2 d \Omega^2
%\end{align}
%In this case, $\alpha$, $a$, and $\beta$ are function of $r$. Also we define 



%obtaining TOV equation. Here, we still have spherical symmetry and staticity so
%\begin{align}
%\frac{d}{dr} ( \sqrt{-g} \rho_0 u^1) &= 0 \nonumber \\
%\frac{d}{dr} ( \sqrt{-g} T^1_0 ) &= 0 \nonumber 
%\end{align}
%where $T_{ab}=(\rho + P) u_a u_b + Pg_{ab}$ or $T_{ab}=\rho_0 h u_a u_b + Pg_{ab}$ where $h = 1+\epsilon+p/\rho_0$ which is specific entalpy

\subsection{Initial Data}
Our initial NS model is approximated by solution of TOV until Schwarzschild radius.

\subsection{Analytic Case}

\end{document}
